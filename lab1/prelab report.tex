\documentclass[12pt]{article}

\usepackage{amsmath, amsthm, amssymb}
\usepackage{enumitem}

\usepackage[margin=2.5cm]{geometry}
\usepackage{setspace}
\usepackage{graphicx}
\graphicspath{ {./images/} }

\renewcommand{\baselinestretch}{1.5}
\title{Lab 1 Building Circuits using Logism Evolution}
\author{Shi Ran 1004793495}


\begin{document}
	\maketitle
	\newpage
	\section*{Part I}\noindent
	1. Below is the gate diagram for $f=xs'+ys$\\
	\includegraphics{part1 1.png}\\
	2. The truth table for the function: \\\\\hspace*{100pt}
	\begin{tabular}{|p{4em}|p{4em}|p{4em}|p{4em}|}\hline
		y & s & x & f \\\hline
		0 & 0 & 0 & 0 \\\hline
		0 & 0 & 1 & 1 \\\hline
		0 & 1 & 0 & 0 \\\hline
		1 & 0 & 0 & 0 \\\hline
		0 & 1 & 1 & 0 \\\hline
		1 & 0 & 1 & 1 \\\hline
		1 & 1 & 0 & 1 \\\hline
		1 & 1 & 1 & 1 \\\hline
	\end{tabular}
	\newpage
	
	\section*{Part II}\noindent
	1. Below is the gate diagram for $f=(a+b)'+cb'$\\\includegraphics{part2 1.png}\\
	2. The truth table for the function: \\\\\hspace*{100pt}
	\begin{tabular}{|p{4em}|p{4em}|p{4em}|p{4em}|}\hline
		a & b & c & f \\\hline
		0 & 0 & 0 & 1 \\\hline
		0 & 0 & 1 & 1 \\\hline
		0 & 1 & 0 & 0 \\\hline
		1 & 0 & 0 & 0 \\\hline
		0 & 1 & 1 & 0 \\\hline
		1 & 0 & 1 & 1 \\\hline
		1 & 1 & 0 & 0 \\\hline
		1 & 1 & 1 & 0 \\\hline
	\end{tabular}\\\\
	3. The diagram in section 1 can be simplified. Perform the following calculation:\\
	\begin{align*}
		f&=(a+b)'+cb'\\
		&=(a'b')+cb'\\
		&=(a'b'+c)(a'b'+b')\tag{by distributive law}\\
		&=(a'b'+c)b'\tag{since b' being true is equvalent to a'b'+b' being true}\\
		&=(a'+c)b'\tag{since b' is satisfied by the second part, it can be removed}
	\end{align*}
	Re-draw the diagram:\\
	\\\includegraphics{part2 3.png}\\
	Note there are five gates in the diagram in section 1, but there are only four gates in this diagram, so this is a cheaper implementation of the design.
	
	
\end{document}
